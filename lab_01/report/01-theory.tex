\chapter{Теоретическая часть}

\section{Равномерное распределение}

Случайная величина $X$ имеет равномерное распределение на отрезке $[a; b]$, если её функция плотности распределения~$f(x)$ имеет вид:
\begin{equation}
	f(x) =
	\begin{cases}
		\frac{1}{b - a}, & \quad x \in [a; b]\\
		0,  & \quad \text{иначе}.
	\end{cases}
\end{equation}

Функция распределения $F(x)$ равна:
\begin{equation}
	F(x) =
	\begin{cases}
		0, & \quad x < a \\
		\frac{x - a}{b - a}, & \quad a \le x \le b \\
		1,  & \quad x > b.
	\end{cases}
\end{equation}

Обозначается $X \sim R[a; b]$.

\section{Нормальное распределение}

Cлучайная величина $X$ имеет нормальное распределение с параметрами~$\mu$~и~$\sigma$, если ее функция плотности распределения~$f(x)$ имеет вид:
\begin{equation}
	f(x) = \frac{1}{\sigma \cdot \sqrt{2\pi}}~~e^{\displaystyle-\frac{(x -
			\mu)^2}{2\sigma^2}}, \quad x \in \mathbb{R}, \sigma > 0.
\end{equation}

При этом функция распределения~$F(x)$ равна:

\begin{equation}
	F(x) = \frac{1}{\sigma \cdot \sqrt{2\pi}} \int\limits_{-\infty}^{x}
	e^{\displaystyle-\frac{(t - \mu)^2}{2\sigma^2}} dt.
\end{equation}

Обозначается $X \sim N(\mu, \sigma^2)$.