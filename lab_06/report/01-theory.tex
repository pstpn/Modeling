\chapter{Теоретическая часть}

\section{Равномерное распределение}

Случайная величина $X$ имеет равномерное распределение на отрезке $[a; b]$, если её функция плотности распределения~$f(x)$ имеет вид:
\begin{equation}
	f(x) =
	\begin{cases}
		\frac{1}{b - a}, & \quad x \in [a; b]\\
		0,  & \quad \text{иначе}.
	\end{cases}
\end{equation}

Функция распределения $F(x)$ равна:
\begin{equation}
	F(x) =
	\begin{cases}
		0, & \quad x < a \\
		\frac{x - a}{b - a}, & \quad a \le x \le b \\
		1,  & \quad x > b.
	\end{cases}
\end{equation}

Обозначается $X \sim R[a; b]$.

\section{Нормальное распределение}

Cлучайная величина $X$ имеет нормальное распределение с параметрами~$\mu$~и~$\sigma$, если ее функция плотности распределения~$f(x)$ имеет вид:
\begin{equation}
	f(x) = \frac{1}{\sigma \cdot \sqrt{2\pi}}~~e^{\displaystyle-\frac{(x -
			\mu)^2}{2\sigma^2}}, \quad x \in \mathbb{R}, \sigma > 0.
\end{equation}

При этом функция распределения~$F(x)$ равна:

\begin{equation}
	F(x) = \frac{1}{\sigma \cdot \sqrt{2\pi}} \int\limits_{-\infty}^{x}
	e^{\displaystyle-\frac{(t - \mu)^2}{2\sigma^2}} dt.
\end{equation}

Обозначается $X \sim N(\mu, \sigma^2)$.

\clearpage

\section{GPSS}

GPSS (General Purpose Systems Simulator~---~ общецелевая система моделирования)~---~язык программирования, используемый для имитационного моделирования систем (в основном, массового обслуживания).
Разработан в 1961 г.
К основным задачам, решаемым с использованием GPSS, относятся:
\begin{enumerate}
	\item моделированием систем массового обслуживания (Q-схемы);
	\item моделирование конечных и вероятностных автоматов (F- и P-схемы);
	\item моделирование сетей Петри (N-, NS-схемы, и т. д.).
\end{enumerate}

Транзакция (сообщение)~---~динамический объект, который создаётся в процессе эксперимента в определённых точках модели, продвигается через блоки и затем уничтожается.
Транзакции перемещаются по блокам модели в направлении увеличения номеров строк программы, описывающих блоки, если только их направление не изменяется под действием специальных блоков.
Все действия над транзакциями выполняются мгновенно с точки зрения модельного времени~---~за исключением явных задержек в специальных блоках, а также ожидания определённых событий.

Пример простой программы на языке GPSS представлен в листинге~\ref{lst:simple.gps}.
\includelisting
	{simple.gps}
	{Пример простой программы на языке GPSS}