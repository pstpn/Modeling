\chapter{Теоретическая часть}

\section{Равномерное распределение}

Случайная величина $X$ имеет равномерное распределение на отрезке $[a; b]$, если её функция плотности распределения~$f(x)$ имеет вид:
\begin{equation}
	f(x) =
	\begin{cases}
		\frac{1}{b - a}, & \quad x \in [a; b]\\
		0,  & \quad \text{иначе}.
	\end{cases}
\end{equation}

Функция распределения $F(x)$ равна:
\begin{equation}
	F(x) =
	\begin{cases}
		0, & \quad x < a \\
		\frac{x - a}{b - a}, & \quad a \le x \le b \\
		1,  & \quad x > b.
	\end{cases}
\end{equation}

Обозначается $X \sim R[a; b]$.

\section{Нормальное распределение}

Cлучайная величина $X$ имеет нормальное распределение с параметрами~$\mu$~и~$\sigma$, если ее функция плотности распределения~$f(x)$ имеет вид:
\begin{equation}
	f(x) = \frac{1}{\sigma \cdot \sqrt{2\pi}}~~e^{\displaystyle-\frac{(x -
			\mu)^2}{2\sigma^2}}, \quad x \in \mathbb{R}, \sigma > 0.
\end{equation}

При этом функция распределения~$F(x)$ равна:

\begin{equation}
	F(x) = \frac{1}{\sigma \cdot \sqrt{2\pi}} \int\limits_{-\infty}^{x}
	e^{\displaystyle-\frac{(t - \mu)^2}{2\sigma^2}} dt.
\end{equation}

Обозначается $X \sim N(\mu, \sigma^2)$.

\section{Принципы управляющей программы}

\subsection{Пошаговый подход}

Заключается в последовательном анализе состояний всех блоков системы в момент $t$ + $\Delta t$ по заданному состоянию в момент $t$.
При этом новое состояние блоков определяется в соответствии с их алгоритмическим описанием с учетом действующих случайных факторов.
В результате этого анализа принимается решение о том, какие системные события должны имитироваться на данный момент времени.
Основной недостаток: значительные затраты машинных ресурсов, а при недостаточном малых $\Delta t$ появляется опасность пропуска события.

\subsection{Событийный принцип}

Характерное свойство модели системы обработки информации: состояние отдельных устройств изменяется в дискретные моменты времени, совпадающие с моментами поступления сообщения, окончания решения задачи, возникновения аварийных сигналов и т. д.
При использовании событийного принципа состояния всех боков системы анализируется лишь в момент появления какого-либо события.
Момент наступления следующего события определяется минимальным значением из списка будущих событий, представляющий собой совокупность моментов ближайшего изменения состояния каждого из блоков.
Момент наступления следующего события определяется минимальным значением из списка событий.
