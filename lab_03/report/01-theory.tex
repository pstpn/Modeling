\chapter{Теоретическая часть}

\section{Методы генерации чисел}

Существует три метода получения последовательности случаных чисел:
\begin{enumerate}
	\item аппаратный;
	\item табличный;
	\item алгоритмический. 
\end{enumerate}

\section{Табличный способ}

Табличный способ подразумевает использование файла (таблицы), содержащего случайные числа.

\section{Алгоритмиический способ}

В качестве алгоритмического способа генерации псевдослучайных чисел был выбран линейный конгруэнтный метод.
Линейный конгруэнтный метод является одной из простейших и наиболее употребительных в настоящее время процедур, имитирующих случайные числа.
В этом методе используется операция $mod(x, y)$, возвращающая остаток от деления первого аргумента на второй.
Каждое последующее случайное число рассчитывается на основе предыдущего случайного числа по формуле~\ref{eq1}.
\begin{equation}
	\label{eq1}
	r_{i+1} = mod(k · r_i + b, M)
\end{equation}
, где $M$~---~модуль (0 < $M$),
$k$~---~множитель (0 <= $k$ < $M$),
$b$~---~приращение (0 <= $b$ < $M$),
$r_0$~---~начальное значение (0 <= $r_0$ < $M$).
Последовательность случайных чисел, полученных с помощью данной формулы, называется линейной конгруэнтной последовательностью.

\newpage

\section{Критерий оценки случайности последовательности}

В качестве критерия для оценки полученных последовательностей чисел была выбрана метрика колебаний последовательности.
В этом критерии будет учитываться:
\begin{enumerate}
	\item среднее значение разницы между соседними элементами (чем больше разброс между	значениями, тем выше считается случайность);
	\item отклонение этой разницы от среднего (меньшая регулярность в изменениях).
\end{enumerate}

Значение критерия вычисляется по формуле~\ref{eq2}.
Чем выше значение метрики, тем более случайной считается последовательность.
\begin{equation}
	\label{eq2}
	m =  meanAbsD/ meanD
\end{equation}
, где $meanAbsD$~---~среднее абсолютное отклонение от среднего,
$meanD$~---~среднее разницы между соседними элементами.
