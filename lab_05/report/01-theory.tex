\chapter{Теоретическая часть}

\section{Схемы модели}

На рисунке~\ref{img:img_01} представлена концептуальная схема модели.
\includeimage
	{img_01}
	{f}
	{H}
	{1\textwidth}
	{Концептуальная схема модели}

В процессе взаимодействия клиентов с информационным центром возможно два режима работы:
\begin{enumerate}
	\item режим нормального обслуживания, при котором клиент выбирает одного из свободных операторов, отдавая предпочтение тому, у кого максимальная производительность;
	\item режим отказа клиенту в обслуживании, при котором все операторы заняты.
\end{enumerate}

\clearpage

На рисунке~\ref{img:img_02} представлена схема модели в терминах систем массового обслуживания (СМО).
\includeimage
	{img_02}
	{f}
	{H}
	{1\textwidth}
	{Структурная схема модели}

\section{Равномерное распределение}

Случайная величина $X$ имеет равномерное распределение на отрезке $[a; b]$, если её функция плотности распределения~$f(x)$ имеет вид:
\begin{equation}
	f(x) =
	\begin{cases}
		\frac{1}{b - a}, & \quad x \in [a; b]\\
		0,  & \quad \text{иначе}.
	\end{cases}
\end{equation}

Функция распределения $F(x)$ равна:
\begin{equation}
	F(x) =
	\begin{cases}
		0, & \quad x < a \\
		\frac{x - a}{b - a}, & \quad a \le x \le b \\
		1,  & \quad x > b.
	\end{cases}
\end{equation}

Обозначается $X \sim R[a; b]$.

\clearpage

\section{Переменные и уравнение имитационной модели}

Эндогенные переменные выглядят следующим образом:
\begin{enumerate}
	\item время обработки задания $i$-ым оператором;
	\item время решения задания на $j$-ом компьютере.
\end{enumerate}

Экзогенные переменные выглядят следующим образом:
\begin{enumerate}
	\item $N_{0}$~---~число обслуженных клиентов;
	\item $N_{1}$~---~число клиентов, получивших отказ.
\end{enumerate}

Вероятность отказа в обслуживании клиента будет вычисляться по формуле:
\begin{equation}
	P_{fail} = \frac{N_{1}}{N_{0} + N_{1}}
\end{equation}

