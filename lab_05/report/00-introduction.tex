\chapter{Задание}

В информационный центр приходят клиенты (пользователи) через интервал времени 10 ± 2 минуты. 
Если все три имеющихся оператора заняты, клиенту отказывают в обслуживании.
Операторы имеют разную производительность и могут обеспечивать обслуживание среднего запроса от пользователя за 20 ± 5, 40 ± 10 и 40 ± 20 ед. времени (минут).
Клиенты стараются занять свободного оператора с максимальной производительностью.
Полученные запросы сдаются в накопитель, откуда выбираются на обработку. 
На первый компьютер~---~от первого и второго операторов, на второй~---~от третьего.
Время обработки запроса в компьютерах~---~15 и 30 минут соответственно.

Смоделировать процесс обработки 300 запросов. 
Определить вероятность отказа.